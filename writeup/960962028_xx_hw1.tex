\documentclass{article} % For LaTeX2e
\usepackage{cos424,times}
\usepackage{hyperref}
\usepackage{url}
\usepackage{graphicx}


\title{ML Genre Classification: \\
\begin{large} Searching for relevant features \end{large} }

\author{\\
\textbf{Rob Whitaker}, 
Physics \\
\texttt{rmw2@princeton.edu} \\
\textbf{Eric Mitchell}, 
Computer Science \\
\texttt{eam6@princeton.edu} \\
}

\newcommand{\fix}{\marginpar{FIX}}
\newcommand{\new}{\marginpar{NEW}}

\begin{document}

\maketitle

\begin{abstract}

\end{abstract}

\section{Introduction}

poopoopeepee

\section{Related Work}

\section{Methods}

\subsection{Feature Selection}
We make a number of observations and assumptions to guide and simplify our analysis.  Importantly, there is nothing comparable about corresponding 20ms frames for any two songs.  We assume that the 30-second excerpts from each song are chosen arbitrarily, if not uniformly, from the song's entirety.  Meaningful ways of comparing frame-level features between songs must then account for the song's architecture.  The simpler way of achieving this amounts to treating all frames as identical and summarizing the distributions of the per-frame values, as we did in our basic classifiers.

\section{Results}

\section{Discussion and Conclusion}

\subsubsection*{Acknowledgments}


\bibliography{ref}

\end{document}
